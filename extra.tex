\documentclass{article}
\usepackage{amssymb, amsfonts, amsmath, amsthm}
\usepackage{hyperref}
\usepackage{enumerate}
\usepackage{graphicx}
\usepackage{ulem}
%\usepackage{url, multicol}


\def\noi{\noindent}%  
\def\CC{\mathbb{C}}%  
\def\PP{\mathbb{P}}%  
\def\NN{\mathbb{N}}%  
\def\RR{\mathbb{R}}%  
\def\ZZ{\mathbb{Z}}% 
\def\QQ{\mathbb{Q}}% 

\def\cC{{\mathcal  C}}%  
\def\cF{{\mathcal  F}}%  
\def\cM{{\mathcal  M}}%  
\def\cS{{\mathcal  S}}%  
\def\cT{{\mathcal  T}}% 

\def\eps{\varepsilon}%  
\def\ge{\geqslant}%
\def\le{\leqslant}%
\def\phi{\varphi}%
\def\i{\subset}
\def\l{\left}
\def\r{\right}
\def\<{\langle}
\def\>{\rangle}
\def\:{\colon}

%operators
\def\Const{\operatorname{Const}}
\def\area{\operatorname{area}}
\def\vol{\operatorname{vol}}
\def\diam{\operatorname{diam}}
\def\codim{\operatorname{codim}}
\def\dim{\operatorname{dim}}
\def\dir{\operatorname{dir}}
\def\dist{\operatorname{dist}}
\def\pack{\operatorname{pack}}
\def\Ric{\operatorname{Ric}}
\def\op{\operatorname}



\begin{document}

\title{Math 429, Extra Credit Problems}
\author{}
\date{}
\maketitle

\begin{center}
{\small These problems are hard and interesting.
The solutions should \\
be presents orally before April 16.
It might improve your score,\\
but should be used for fun. Only the first solution will be graded.}
\end{center}
\thispagestyle{empty}




\noi $1$. 
Find three disjoint open sets on the real line 
that have the same nonempty boundary. 

\ 

\noi $2$. Construct a continuous function 
$f\:[0,1]\rightarrow [0,1]$ such that $f$ takes every value in $[0,1]$ 
an infinite number of times.

\ 

\noi $3$.
Describe all the homeomorphisms from the \href{http://en.wikipedia.org/wiki/Sierpinski_triangle}{Sierpinski triangle} to it self.

\ 

\noi $4$. Prove that $\RR^3\backslash S^1$ is homeomorphic to $\RR^3\backslash (\ell\cup \{p\})$, where $\ell$ is a straight line and $p\not\in\ell$ is a point in $\RR^3$.


\ 


\noi $5.$ Show that any nonempty open star-shaped set in $\RR^2$ is homeomorphic to the open disc.

\ 

\noi t.b.c.


\end{document}

\ 


\noi $8.$ Construct two functions $\RR\to\RR$, one is closed but not continuous, and the other is open but not continuous.

\ 

\noi $9.$
Construct a bounded open set $\Omega$ in $\RR^2$
such that its boundary $\partial \Omega$ is totally path-diconnected;
that is, no pair of distinct points $x,y\in\partial \Omega$ can be connected by a path in $\partial \Omega$.

\ 

\noi $10.$ Let
\[A=\{\,(x,y)\in\RR^2\mid x,y\in\QQ\,\}
\quad\text{and}\quad
B=\{\,(x,y)\in\RR^2\mid x,y\notin\QQ\,\}.\]
Show that $A\cup B$ is path connected.

\ 

\noi $11.$ Show that any connected finite space is path connected.

\ 

\noi $12.$ Let $X$ be a path connected space, consider space 
$$Z=X\times X/\sim,$$
where $(x,y)\sim(y,x)$.
Prove that $\pi_1(Z)$ is abelian.

\ 

\noi $13.$ Prove that any two spaces with the same homotopy type 
can be embedded as
deformation retracts in the same topological space.

\ 

\noi $14.$ Let $X$ be a topological space, $U, V \subset  X$ two open subsets. 
Prove that if
$U \cup V$ and $U \cap V$ are simply connected, then so are $U$ and $V$.

\ 

\noi $15.$ Let a topological space $X$ be the union of two open path-connected sets $U$ and $V$. 
Prove that if $U\cap V$ has at least three connected components, then the fundamental group of $X$ is non-Abelian.

\ 

\noi $16.$ Construct two covering maps $f\:X\to Y$ and $g\:Y\to Z$ such that the composition $gf\:X\to Z$ is not a covering map.

\ 

\noi $17.$ Construct a finite topological space with fundamental group isomorphic to~$\ZZ_2$.

\ 

\noi $18.$ 
Suppose $A$ and $B$ are two subset of $\RR$.
Suppose that $A\cup B$ and $A \cap B$ are connected, but $A$ and $B$ are not connected.

\ 

\noi $19.$
Classify all topological spaces containing unique nowhere dense subset.


\ 

\noi $20.$ Let $\mathcal{T}$ be the topology on $\RR$ induced by the maps $x\mapsto e^{i\cdot t\cdot x}$ for all $t\in \RR$.
Show that $(\RR,\mathcal{T})$ is not metrizable.




\ 

\noi $7$. Find two compact subsets $A,B\i\RR^2$ such that 
$A$ is not homeomorphic to $B$ but $A\times[0,1]$ is homeomorphic to $B\times[0,1]$.


\ 

\noi $4$. Prove that the set of rational numbers 
$\QQ$ is not an intersection of a countable collection of open sets in $\RR$.

\ 


\noi $1$. Construct a topology $\mathcal{T}$ on $\RR$ such that a function $f\:\RR\to\RR$ is nondecreasing if and only if it is continuous for the topology $\mathcal{T}$.

\ 

\noi \sout{$\,3.$} \textit{(solved)} How many pairwise distinct sets can one obtain
from of a single set by using the operators closure and interior?

\ 

\q1. Локально компактное и хаусдорфово пространство~--- регулярно.

\medskip

\q2. Метрический компакт не изометричен собственному подмножеству.

\medskip

\q3. В хаусдорфовом компакте вложенное пересечение связных замкнутых множеств~--- связно.

\medskip

\q4. Тихоновское произведение связных пространств связно.

\medskip

\q5. Тихоновское произведение регулярных пространств регулярно.

\medskip

\q6. Пример: секвенциальное замыкание не секвенциально замкнуто.

\medskip

\q7. Канторово множество гомеоморфно $\{0, 1\}^\N$.

\medskip

\q8. Гомеоморфны ли букет счетного числа окружностей и гавайская серьга?

\medskip

\q9. Гомеоморфны ли гребенка 
$$
(\{0\}\times [0,1])\cup (\{1/n~|~n\in \mathbb {N} \}\times [0,1])\cup ([0,1]\times \{0\})
$$ 
и урезанная гребенка 
$$
(\{0\}\times \{0,1\})\cup (\{1/n~|~n\in \mathbb {N} \}\times [0,1])\cup ([0,1]\times \{0\})?
$$

\medskip

\q10. Отображение из компакта в себя, уменьшающее расстояния, имеет неподвижную точку.

\medskip

\q11. Пусть $A$~--- замкнутое подмножество метрического пространства $X$.
Тогда всякую непрерывную функцию $$f\colon A\to [-1; 1]$$ можно продолжить на всё пространство $X$.

\medskip

\q12. Если две метрики задают одну топологию, то их сумма задает ту же топологию.

\medskip

\q13. Фактор хаусдорфова компакта по замкнутому отношению~--- хаусдорфов.

\medskip

%\q14. Пространство склеивается из элементов фундаментального покрытия.

\medskip

\q16. $\R^3$ минус окружность гомеоморфно $R^3$ минус прямая минус точка.

\medskip

\q18. $\epsilon$-окрестность связного множества на плоскости линейно связна.

\medskip

\q19. Если есть счетная база, то из любой базы можно выделить счетную.

\medskip

\q20. Пример: пересечение двух компактных множеств некомпактно.

\medskip

\q21. Любая окрестность компакта в метрическом пространстве содержит $\varepsilon$-окрестность.

\medskip




\q22. Метрическое пространство компактно и замыкание любого открытого шара есть замкнутый шар
того же радиуса с тем же центром.
Тогда любой шар связен.

\medskip

\q23. Связно ли $(\Q \times \Q) \cup ((\R\setminus \Q) \times (\R\setminus \Q) )$?

\medskip

\q24. График непрерывного отображения в хаусдорфово пространство замкнут.

\medskip

\q25. Если у отображения в компактное пространство график замкнут, то оно непрерывно.

\medskip

\q26. Букет счетного набора отрезков не метризуем.


\medskip

\q27. График функции из компактного пространства в хаусдорфово замкнут. Докажите, что функция непрерывна.

\medskip

\q28. Какие из пар пространств $\R/\Z$ (имеется в виду, что $x\sim y \Leftrightarrow x-y\in\Z$), $S^1$, $[0,1)$ гомеоморфны, а какие нет?
 
\medskip

\q29. тремерная сфера = два полнотория
 
\medskip

\q30. факторпространство плоскости по лучу гомеоморфно открытый шар+точка на границе

\medskip
 
{\bf Определение.}
{\it Норма} в векторном пространстве $V$  над полем $\R$~--- это функционал $p:V\to [0,\infty)$, обладающий следующими свойствами:

1.
$ p(x)=0 \Leftrightarrow x=0$; 

2.  $p(x+y)\leq p(x)+p(y)$,  $ \forall x,y\in V$ (неравенство треугольника);

3. $p(ax)=|a|p(x)$  $ \forall x\in V,\forall a\in\R$
 
\medskip
 
 
\q31. Пусть $d \colon \R^n \times \R^n \to R$~--- метрика на $V$, инвариантная
относительно параллельных переносов. 
Предположим, что $d$ удовлетворяет условию
$d(\lambda x, \lambda y) = |\lambda| d(x, y)$
для всех $\lambda\in \R$. 
Докажите, что $d$ получается из нормы $\rho \colon \R^n \to R$ по
формуле $d(x, y) = \rho(x - y)$.



\medskip

\q32. Пусть $f_i$~--- последовательность $C$-липшицевых 
функций, поточечно сходящаяся к $f$. Докажите, что $f$ непрерывна.

\medskip


\q33. Пусть $\mathcal M$~--- множество всех замкнутых ограниченных подмножеств метрического пространства $X$ с метрикой хаусдорфа. 
Докажите, что если $X$ компактно, то и $\mathcal M$ компактно. 

\medskip

\q34. 
%a) Докажите, что множество плоских выпуклых многоугольников в метрике Хаусдорфа линейно связно.

b) Верно ли аналогичное утверждение для произвольных многоугольников?

\medskip

\q35. Приведите пример полного, не локально компактного метрического пространства.

\medskip

\q36. Постройте континуальное нигде не плотное подмножество в отрезке
$[0, 1]$ с естественной топологией.

\medskip

\q37. Пусть $X$~--- хаусдорфово пространство. 

a) Докажите, что у любой последовательности есть не более одного предела.

b) Верно ли обратное (т. е. вытекает ли хаусдорфовость из единственности предела)? 

\medskip

\q38. Существует ли компактное хаусдорфово 
неметризуемое топологическое пространство?

\medskip

\q39. Верно ли, что произведение нормальных пространств нормально?
%Написать про плоскость Зоргенфрея...

\medskip

\q40. Пусть $X$~--- компактное метрическое пространство, 
$f\colon X\to X$~--- изометрия, т.е.  $d(f(x),f(y)) =d(x,y)$ для всех $x,y\in X$.  
Докажите, что $f$ является гомеоморфизмом.

\medskip

\q Канторово множество. Определим $K_0=[0,1]$. $K_{n+1}$ получается 
из $K_n$ выкидыванием из $K_n$ средней трети в каждом отрезке. То что получится
в результате (в пределе?) называется Канторовым множеством.
\medskip

\q41. a) Что это значит? Что за предел?

b) Докажите, что канторово множество замкнуто, а его внутренность пуста.

c) Докажите, что в канторовом множестве содержатся те и только те числа от 0 до
1, которые в троичной системе счисления можно записать без цифры 1. (Зачем такая
странная формулировка "можно записать, ..."?)

d) Докажите, что его длина равна 0. (А что бы это формально означало?)

e) Можно ли как-нибудь изменить, чтобы длина перестала быть равной 0?

f) Канторово множество компактно.

g) Компоненты связности канторова множества --- точки.

e*) Докажите, что компактное пространство без изолированных точек со второй
аксиомой счетности, в котором компоненты связности точки гомеоморфно 
канторовому множеству.

\medskip

\q42. Докажите, что топологическое пространство $(X;\Omega)$ связно тогда и только тогда, когда
любая непрерывная функция $f\colon X\to K$, где $K$ канторовское множество, является постоянной.

\medskip

\q43. Докажите, что любое компактное сепарабельное хаусдорфово топологическое 
пространство без изолированных точек содержит замкнутое подмножество гомеоморфное канторовскому множеству.

\medskip

\q44. Докажите, что множество $[0;1]\times [0;1]\setminus K\times K$,
где $K$ канторовское множество, линейно связно.

\medskip

\q45. Рассмотрим самую слабую топологию $\Omega$ на множестве $\R$ вещественных чисел, в которой  
для любого вещественного  $t\in \R$ отображение  $x\to (\cos tx, \sin tx)\in \R^2$ непрерывна.
Докажите, что $\Omega$ не метризуема. 

\medskip

\q46. Докажите, что дополнения любых двух всюду плотных счетных подмножеств в $\R^2$ гомеоморфны.

\medskip
\q47. Если $\{K_n\}$ ($n$ от 1 до бесконечности)~--- убывающая последовательность компактных непустых связных подмножеств хаусдорфова пространства, то их пересечение непусто и связно.

\medskip

\q48. Докажите, что плоскость с удаленной точкой гомеоморфна кольцу $\{a < x^2 + y^2 < b\}$, где $0 < a < b$.

\medskip

\q49. Докажите, что пространство целых чисел негомеоморфно пространству рациональных чисел (с топологиями, индуцированными из $\R$).

\medskip
\q50. Постройте пример: секвенциальное замыкание не совпадает с замыканием.

    



\end{document}

% \q1. Пусть $f$~--- функция, удовлетворяющая следующим трем свойствам
% \begin{itemize}
%     \item[1)] $f(0)=0$; 
%     \item[2)] $f$ монотонно возрастает; 
%     \item[3)] $f(x+y)\leq f(x)+f(y)$ для любых $x,y\in \mathbb R$.
% \end{itemize}
% Докажите, что $f(d)$ является метрикой, если $d$ --- метрика.

\q1. Является ли связным, линейно-связным, компактным множество 
$$([0,1]\times[0,1])\cap((\Q\times \Q)\cup(\R\setminus\Q\times\R\setminus \Q))?$$

\medskip

\q6. Пусть центрально-симметричное выпуклое 
множество $B \subset \R^n$ не содержит лучей и пересекается с каждым лучом $\{\lambda v : \lambda > 0\}$. Рассмотрим функцию
$$\nu \colon v \to \sup\{\lambda \in \R_+ \mid \lambda^{-1} v \notin B\}.$$
Докажите, что это норма на $V$. Докажите, что все нормы 
получаются таким образом.
\medskip

\q8. Пусть $\mathcal M$~--- множество всех замкнутых ограниченных подмножеств метрического пространства $X$ с метрикой хаусдорфа. 
Докажите, что если $X$ полно, то и $\mathcal M$ полно. 

\medskip

\q16. Пусть дано метризуемое счетное связное 
пространство $X$. Докажите, что $X$~-–– это точка.

\medskip

\q10. a) Докажите, что множество плоских выпуклых многоугольников в метрике Хаусдорфа линейно связно.

\medskip

\q13. Приведите пример счетного хаусдорфова пространства без счетной базы.



 

(\{0\} \times [0,1] ) \cup (K \times [0,1]) \cup ([0,1] \times \{0\})












